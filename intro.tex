\section{Introduction}

Background(一段) 

\begin{itemize}
\item 1. GPU性能评估的重要性;
\item 2. GPU性能评估需要考虑的问题,准确性,指导性等等。
\end{itemize}

Motivation(一段) 

\begin{itemize}
\item 1. 监控工具可以监控的指标有非常多种,需要熟练的技术知识才能分析这些指标;
\item 2. 指标太多不利于直接发现最关键的性能瓶颈,
需要一定的经验知识作为辅导;
\item 3. 数学模型的建立有助于评估性能,但是精确的模型的建立及理解都很困难,对模型建立者和使用者都需要对GPU架构有深刻理解。
\end{itemize}

Introduction(两段) 

监控工具得到的指标非常多,往往一个指标或者多个指标集可以指向一个具体的性能问题。而不同的GPU程序又必然有一个影响最大的性能瓶颈。所以能够快速定位影响最大的性能瓶颈非常关键。决策树的分析模型是机器学习中基于信息论的分类模型,更加专注于信息对整个数据集的影响,因此由决策树决定需要考虑的指标,非常适合解决该问题。

获取影响最大的指标集后,极大地缩小了需要考虑的范围。如果继续采用决策树的分类性能瓶颈和优化方案具有一定的误差,这与机器学习特性有关。传统的数学模型准确度往往比机器学习算法准确率要高,由于需要考虑的指标集已经很小了,因此通过理论模型分析这些指标更为恰当。分析方法从三个层面考虑:应用层面(包括应用在计算需求、数据排列上的处理),系统层面(包括指令调度、资源管理和分配等),硬件层面(包括通信带宽、资源特性等)。结合小范围的数据集和三个层面,本文模型可以快速定位并准确提出性能瓶颈与优化方案。

Contribution(三点) 

\begin{itemize}
\item 1. 提供了一种结合决策树模型和理论分析模型的GPU应用性能分析模型,可以更加准确地定位性能瓶颈并提出优化方案。
\item 2. 提供了一种基于决策树的分析算法,在GPU应用程序的监控数据中,由决策树决定影响最大的指标集并做进一步的理论分析。决策树的方法可以快速地决策GPU应用程序的瓶颈,缩小需要考虑和分析的范围。决策树不仅基于microbenchmark,还可以基于上层数据处理系统的benchmark训练,适用范围广。
\item 3. 基于决策树的分析结果,提出了面向应用、系统和硬件的三层理论分析模型,进一步定位决策树得出的瓶颈结果。理论分析模型可以在小范围内更加准确地确定GPU应用瓶颈并提出优化方法。
\end{itemize}

Paper(一段) 

第II章,相关工作;第III章:背景;第IV章,决策树;第V章,分析模型;第VI章,实验;第VII章,结论。