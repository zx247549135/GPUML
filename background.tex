\section{Background}

\subsection{Features in ML}

% 特征工程是机器学习应用中重要的两个模块之一,特征用来评价一条记录的某个方面,特征的好坏决定了模型的准确性。
% 机器学习对特征的要求是各个特征之间的关联性很小,尽量相互独立,能够反映数据的特点。例如,GPU应用的一条监控数据,包括三个特征:Kernal函数的程序长度Code_Line,编译后的Kernal函数指令条数Instruction_Line,SM配置的Block的维度Block_X,Code_Line和Instruction_Line有明显的决定关系,两者关联度很高,在一般情况下Code_Line和Instruction_Line成线性关系,作为两个特征并不合适,而Block_X和Instruction_Line之间没有直接关系,可以作为两个重要特征。

Features in ML.

\subsection{Decision Model in ML}

% 机器学习的训练模型中有两大类,基于线性拟合的模型(Regression Model),如逻辑回归(Logistics Regression),神经网络(Neural Network),深度学习(Deep Learning)等;基于信息论的决策模型(Decision Model),如决策树(Decision Tree),随机森林(Random Forest)等。
% 机器学习模型的训练本质上是对特征的处理:Regression Model通过多次迭代,更新模型中各项特征的权重矩阵,最终权重矩阵收敛到一个稳定值;Decision Model则通过多次划分,每次选择最优的特征来划分数据集,形成树形模型。

% 在对新的数据做分类预测时,Regression Model直接利用权重矩阵与新数据的特征值矩阵相乘,预测新数据的类别,而Decision Model则按照树形模型依次考虑新数据的每个特征值所属类别。因此,Decision Model更适合于分析预测的过程,而且基于信息论的决策过程也可以保证科学性。
% 在Decision Model上最基础的实现是Decision Tree,基于信息划分理论实现Decision Tree。在Decision Tree基础上发展的Random Forest则通过多颗树形成森林,然后对结果做综合预测,进一步提高机器学习模型的准确性。

% 在GPU应用的监控数据中,我们需要考虑监控数据的重要性,首先需要做的是将监控的metric转变为Feature,形成一条记录。然后选择机器学习的模型来决定重要的Feature,很明显Decision Model更适合于中间过程的分析。由于机器学习模型极大的方便了用户,屏蔽了非常重要的技术分析细节,准确率往往比传统的数学模型要低。为了保证GPU应用分析的重要性,结合传统的数学分析模型更为重要。所以使用Decision Tree而不是Random Forest,对使用理论模型分析性能的情况更为高效。

Decision model in ML.

\subsection{GPU}

GPU architecture.