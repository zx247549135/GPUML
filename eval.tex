\section{Evaluation}

\subsection{Configuration}

We use four nodes as workers and one node as the master in the experiments. Each node has two eight-core Xeon-2670 CPUs and 64GB memory. The file system is mount on one SAS disk, running RedHat Enterprise Linux 5 (kernel 2.6.18). The JDK version is 1.7.0 for Spark 1.6 and Spark Job Server 0.6.2. We count not only the execution time of each application, but also the details of runtime situations.

\begin{table}[!t]
\small
\centering
\caption{Function APIs in each Application}
\begin{tabular}{ c | c | c | c }

\hline
\textbf{App} & \textbf{Stages} & \textbf{Function API(s)} & \textbf{Cache} \\
\hline
Grep & 1 & \textit{filter} & No \\
\hline
WC & 2 & \textit{flatMap} \& \textit{reduceByKey} & No \\
\hline
Sort & 3 & \textit{distinct} \& \textit{sortByKey} & No \\
\hline
PR & N & \textit{groupByKey} \& \textit{map} \& \textit{reduceByKey} & Yes \\
\hline

\hline
\end{tabular}
%\vspace{1mm} 
%\vspace{-8mm}
\label{table:app}
\end{table} 

We choose four typical benchmark applications in Spark to evaluate the performance: Grep, Sort, WordCount(WC), and PangeRank(PR). Grep has only one stage: it filters the records that do not satisfy the conditions. WC has two stages: it counts the number of each key in the input file. Sort has three stages: it sorts all records by key in the input file. While PR is a typical iterative computations, and one of its most important features is that it will cache data in memory. We choose these four benchmarks because these applications contains different function APIs, as shown in Table~\ref{table:app}. For Sort  and WC, the datasets are produced by HiBench Random Writer with 1B unique key numbers. The size of input dataset is 30GB in Sort, and 50GB in WC. For Grep and PR, we use the real graphs: webbase-2001 (30GB)~\cite{boldi:webgraph} to evaluate the performance of MURS. Key-value pairs of each record in all input dataset have the similar size. We tun the size of heap size to evaluate different memory pressure. Garbage collection time is used to measure the memory pressure. These applications are grouped to evaluate different scenes and submitted to Spark Job Server together.

\subsection{Memory pressure without caching}

Most current data processing systems are designed based on MapReduce, but only parts of them provide the in-memory computing model that caches data in memory to speed up the system. Thus, we first evaluate these applications without caching data in memory to stand for common frameworks working with key-value pairs, such as Hadoop and Hive.

We choose three applications: Sort, WC, and Grep. These applications have no caching operation. Each application has similar implementation in MapReduce. Grep reads data from disk and filter these records which satisfy the given conditions. Most data objects are temporary. Shuffle buffers in Sort and WC are data objects with long lifetime. Thus tasks in Sort and WC are clear to heavy tasks in MURS. The results of each submission are shown in Figure~\ref{fig:pressurewithoutcache}. The best improvement of MURS can be 1.8x to 2.9x compared to Spark in each evaluation. The reduction of garbage collection contributes to most improvement.

\begin{figure*}[!t]
\centering
\subfigure[Sort+Grep]{
\label{fig:subfig:sort-grep}
\includegraphics[width=0.3\textwidth]{sort-grep.pdf}}
%\hspace{-3ex}
\subfigure[WC+Grep]{
\label{fig:subfig:wc-grep}
\includegraphics[width=0.3\textwidth]{wc-grep.pdf}}
%\hspace{-1.9ex}
\subfigure[Sort+WC+Grep]{
\label{fig:subfig:sort-wc-grep}
\includegraphics[width=0.3\textwidth]{sort-wc-grep.pdf}}
%\hspace{-1.9ex}
%\subfigure[Active tasks]{
%\label{fig:subfig:activetasks}
%\includegraphics[width=0.3\textwidth]{active-task.pdf}}
%\vspace*{-5mm}
\caption{Results of memory pressure without caching}
%\vspace*{-4mm}
\label{fig:pressurewithoutcache}
\end{figure*}

When two applications are submitted to the server, MURS works better in Sort and Grep. Comparing Sort to WC, heavy memory pressure occurs in different phase. The \textit{sort} operation is implemented in the \textit{read phase} of tasks. Thus the tasks in the last stage of Sort suffer heavy memory pressure during their read phase. However, \textit{reduce} operation in WC is realized in the \textit{write phase} of tasks. Heavy memory pressure is resulted in by the write phase of tasks in the first stage. What's more, massive temporary data objects are produced by the function API \textit{flatMap} before the write phase. They may be currently alive in the heap during the write phase in WC. Although tasks in Sort belong to linear and tasks in WC are sub-linear based on the memory usage model, MURS suspends more tasks in WC as the heap size in WC is occupied by the temporary data objects, as shown in Figure~\ref{fig:active-task}. And these temporary data objects result in frequenter garbage collection. We can see that suspending in MURS is determined not only by the memory usage models, but also the current usage of heap. 

While three applications are submitted, out-of-memory (OME) is thrown by the server. Although the live shuffle buffer is smaller than those sever with two applications, some other data objects cost much heap space, such as the recording of living applications, the handler of disk writing, and the global configurations of Spark. Spark provides spill to avoid the shortage of memory. However, it cannot completely avoid the OME.

MURS mitigates memory pressure in each application and we find that the performance slows down tardily along with the heap size. The result shows that the cost of garbage collection is stable. As MURS suspends the heavy tasks, the tardily increasing cost in execution time is the computation time actually.  

\begin{figure}[!t]
\centering
\includegraphics[width=0.35\textwidth]{active-task.pdf}
%\vspace{-2mm}
\caption{The minimum active tasks in each server}
%\vspace{-4mm}
\label{fig:active-task}
\end{figure}

\subsection{Memory pressure with caching}

Some frameworks provide caching mechanism for in-memory computation, such as Spark and Flink. Although the in-memory caching data speed up the execution of a job, some works~\cite{bu:bloat, nguyen2015facade} show that it results in greater memory pressure because the caching data live along with the job. Tracing these data is expensive and there is less accessible memory space for execution.

We choose PR and WC as the submitted applications. PR is an iterative application and we run 5 iterations in our experiments. PR caches intermediate data in memory after the first iteration. WC is submitted at the second iteration. We adjust the heap size to show the performance of MURS, while the input dataset is 30GB, as shown in Figure~\ref{fig:cache-total}. When the heap size is less than 17GB, Spark in server will throw Out-of-Memory error (OME). MURS can provide server although the heap size is reduced to 15GB. While they are both working, MURS improves the performance by up to 23.4\%, and the mitigation of memory pressure achieve a 65.4\% improvement.

\begin{figure}[!t]
\centering
\includegraphics[width=0.45\textwidth]{cache-total.pdf}
%\vspace{-2mm}
\caption{The Execution and GC Time of PR and WC in Server}
%\vspace{-4mm}
\label{fig:cache-total}
\end{figure}

Less heap size in each node means more memory pressure in the server. When the heap is exhausted, tasks will throw the OME. With the caching data in memory, Spark for server throws the OME as the heap size of each node is 17GB. However, MURS is committed to mitigate the heavy memory pressure and still working well although the heap size of each node reduces to 15GB. MURS suspends heavy tasks to reduce the active tasks, thus running tasks can utilize more memory. Figure~\ref{fig:cache-peak} shows that the peak memory usage of all tasks in MURS persists to be larger than Spark. And the minimum tasks in MURS is reduced to ensure the available memory for running tasks. In other words, MURS has stronger scalability than Spark for server.

\begin{figure}[!t]
\centering
\includegraphics[width=0.45\textwidth]{cache-peak.pdf}
%\vspace{-2mm}
\caption{The peak usage memory of tasks and minimum active tasks}
%\vspace{-4mm}
\label{fig:cache-peak}
\end{figure}

\subsection{Potential starvation in MURS}

Note that the mitigation of memory pressure leads to the delay in computation. This will lead to potential starvation in suspended tasks. Fortunately, MURS solve this problem in the application level. We take the same configurations in the previous subsection. While PR and WC are submitted to the server, tasks of PR have the same \textit{write phase} as tasks of WC. But they contain more function APIs. Thus tasks of PR are mostly classified as heavy tasks. They will be suspended when the memory pressure is heavy. The execution time of each application is shown in Figure~\ref{fig:cache-prwc}. The result shows no starvation in PR, and PR is also benefit from MURS and the performance is improved by up to 24.4\%. The performance in WC which has light tasks also achieves a 29.8\% improvement. 

\begin{figure}[!t]
\centering
\includegraphics[width=0.35\textwidth]{cache-prwc.pdf}
%\vspace{-2mm}
\caption{The execution time of each application}
%\vspace{-4mm}
\label{fig:cache-prwc}
\end{figure}

MURS provides FIFO algorithm when it resumes the suspended tasks. This avoids the long wait of these tasks. Otherwise, suspending these tasks is to mitigate the memory pressure, not only for the current running light tasks but also the next running heavy tasks. In the application level, transient suspending and lighter memory pressure are actually trade-off. 

\subsection{Avoidance of spill}

MURS sets the red value to avoid spill when memory pressure is heavy in these systems who provide spill. We take the evaluation in the last section, which runs PR and WC in the Spark for service, to measure the spill tasks. MURS estimates the size of required memory space for each task, and ensures that running tasks can complete with the remaining memory space. As the estimation is inaccurate to some extent, we find that fewer tasks will still spill in MURS, as shown in Table~\ref{table:spill}. There are no spill tasks in WC in MURS and the spill tasks in PR decreased from 32\% to 2.5\% in comparing to Spark.

\begin{table}[!t]
\small
\centering
\caption{Spill Tasks in MURS and Spark}
\begin{tabular}{ c | c | c | c | c | c | c | c }

\hline
\multirow{2}{*}{} & \multirow{2}{*}{\textbf{App}} & \multicolumn{3}{| c |}{\textbf{Spill Percentage}} & \multicolumn{3}{| c }{\textbf{Spill Size (MB)}} \\
\cline{3-8}
 & & total & spill & percent & min & mid & max \\
\hline
\multirow{2}{*}{Spark} & WC & 1000 & 91 & 9\% & 0 & 0 & 710 \\
\cline{2-8}
 & PR & 1500 & 480 & 32\% & 310 & 367 & 439 \\
\hline
\multirow{2}{*}{MURS} & WC & 1000 & 0 & 0\% & - & - & -  \\
\cline{2-8}
 & PR & 1500 & 37 & 2.5\% & 0 & 0 & 458 \\
\hline

\hline
\end{tabular}
%\vspace{-2mm}
 
%\vspace{-6mm}
\label{table:spill}
\end{table}

We should acknowledge that error exists in the avoidance of spill. The sampler in MURS counts two important metrics of one task: the percentage of processed records in total records, and the current allocated memory space for this task. We can quickly get the required memory space based on these two metrics. When memory pressure reaches the red value, we suspend parts of tasks and leave enough remaining memory space for running tasks. As the estimate is based on sampling, error exists in the estimated size of allocated memory space, especially when the value in some key-value pairs is a collection. Different records have different sizes because the number of values inside a collection is different, such as \textit{groupByKey} in PR. Some hot keys may result in substantial error. Thus, it can be accepted that there are even fewer spill tasks in MURS.

















\begin{comment}

\subsection{Configuration}

We use four nodes as workers and one node as the master in the experiments. Each node has two eight-core Xeon-2670 CPUs and 64GB memory. The file system is mount on one SAS disk, running RedHat Enterprise Linux 5 (kernel 2.6.18). The JDK version is 1.7.0 for Spark 1.6 and Spark Job Server 0.6.2. We count not only the execution time of each application, but also the details of all tasks. Some important configurations are set to be the same in both MURS and Spark, as shown in Table~\ref{table:config}. The memory of each executor is set to be 15GB.

\begin{table}[!t]
\small
\centering
\caption{Important Configurations in MURS and Spark}
\begin{tabular}{ c | c | c }

\hline
\textbf{Configurations} & \textbf{Default} & \textbf{Value} \\
\hline
spark.serializer & No & KryoSerializer \\
\hline
spark.cores.max & No & 48 \\
\hline
spark.shuffle.compress & Yes & true \\
\hline
spark.shuffle.manager & Yes & sort \\
\hline
spark.memory.fraction & Yes & 0.75 \\
\hline
spark.memory.storageFraction & Yes & 0.5 \\
\hline

\hline
\end{tabular}
%\vspace{-2mm} 
%\vspace{-6mm}
\label{table:config}
\end{table} 

We choose four typical benchmark applications in Spark to evaluate the performance: Grep, Sort, WordCount(WC), and PangeRank(PR). Grep has only one stage: it filters the records that do not satisfy the conditions. WC has two stages: it counts the number of each key in the input file.  Sort has three stages: it sorts all records by key in the input file. While PR is a typical iterative computations, and one of its most important features is that it will cache data in memory. We choose these four benchmarks because these applications contains all three models. We will show the details later. For Grep, Sort, and WC, the datasets are produced by HiBench Random Writer with different unique key numbers (1M and 1B), both with the size of 50GB. And for PR, we use the real graphs: webbase-2001 (30GB)~\cite{boldi:webgraph} to evaluate the performance of MURS. Key-value pairs of each record in all input dataset have the similar size, thus we can simply analyze the memory usage model by function API. These applications are grouped to evaluate different scenes and submitted to Spark Job Server together.

\subsection{Memory pressure without caching}

Most current data processing systems are designed based on MapReduce, but only parts of them provide the in-memory computing model that caches data in memory to speed up the system. Thus, we first evaluate these applications without caching data in memory to stand for common frameworks working with key-value pairs, such as Hadoop and Hive.

We choose Grep, Sort, and WC to form the benchmark set. The function APIs in each job are shown in Table~\ref{table:grep-sort-wc}. Grep has no shuffle operation. WC and Sort jobs are split by the shuffle operations. Tasks in the first stage of WC result in more memory pressure than second stage, because most records are aggregated in the write phase. In contrast, Sort actually sorts keys in the read phase of tasks in the third stage. Jobs are simultaneously submitted to Spark Job Server. The execution time of each job is decreased by 20\% to 30\%. We count the execution time and garbage collection time of all tasks in each stage, as shown in Figure~\ref{fig:grep-sort-wc-gc}. Tasks with constant model (Stage 2-3) all have a short execution time. They also benefit from MURS and the execution time is decreased by 59\%, although the garbage collection time of them has less degradation. Execution time of tasks with sub-linear model (Stage 0) is decreased by 54\% and the reduction comes all from the mitigation of garbage collection. Execution time of tasks with linear model (Stage 5) is reduced by 50\%, and the garbage collection time is decreased by 58\%.

\begin{table}[!t]
\small
\centering
\caption{Function APIs in each Application without caching}
\begin{tabular}{ c | c | c | c }

\hline
\textbf{Application} & \textbf{Stage ID} & \textbf{Function API} & \textbf{Model}\\
\hline
\multirow{2}{*}{WC} & Stage 0 & \textit{reduceByKey} & sub-linear  \\
\cline{2-4}
 & Stage 1 & \textit{reduceByKey} & sub-linear \\
\hline
Grep & Stage 2 & \textit{filter} & constant \\
\hline
\multirow{3}{*}{Sort} & Stage 3 & \textit{flatMap} & constant \\
\cline{2-4}
 & Stage 4 & \textit{sortByKey} & linear \\
\cline{2-4}
 & Stage 5 & \textit{sortByKey} & linear \\
\hline

\hline
\end{tabular}
%\vspace{1mm} 
%\vspace{-8mm}
\label{table:grep-sort-wc}
\end{table}

%We chose Grep, Sort, and WC to group the submitted jobs. The function API in Grep is \textit{filter}, which does not distinguish the key and process each record. It just judges whether the record satisfies given conditions and writes it to disk; thus, all data objects are temporary data objects and the model belongs to the constant model. The function API in Sort is \textit{sortByKey}, which must shuffle all records and partition the records in order by key. Each record will be saved in shuffle buffer before writing to disk, thus it belongs to the linear model. The function API in WC is \textit{reduceBykey}, which distinguishes the key and aggregates the value. It is a typical sub-linear model. From the model, we find that tasks in Sort may have more impact on memory pressure. The result shows that the execution time of WC is decreased by 20\%, Grep is decreased by 15\%, and Sort is decreased by 13\%. We take the execution time and garbage collection time of tasks to analyze the improvement, as shown in Figure~\ref{fig:grep-sort-wc-gc}.

\begin{figure}[!t]
\centering
\includegraphics[width=0.4\textwidth]{grep-sort-wc-gc.pdf}
%\vspace{-2mm}
\caption{The Execution and GC Time of Tasks in Sort, Grep and WC}
%\vspace{-4mm}
\label{fig:grep-sort-wc-gc}
\end{figure}

Tasks with constant model in Stage 2 and Stage 3, along with tasks with sub-linear model in Stage 1, all benefit from MURS although the memory pressure is light. This is due to the parallelism in writing to disk, as Spark writes intermediate data in shuffle write phase to disk for checkpoint. Note that when these tasks are running, tasks in Stage 5 have not been launched. MURS is triggered when the proportion of used heap first touch the yellow value, it suspends some tasks in WC and decreases the parallelism in writing to disk. With enough bandwidth of writing to disk, these tasks still execute fast. We find that tasks in Stage 4 benefit less from the parallelism in writing to disk, because they are linear models and suspended by MURS when the proportion of used heap touches the yellow value. 

The reductions of Stage 0 and Stage 5 prove the mitigation on memory pressure through MURS. The garbage collection time in WC has a greater degradation than Sort. While WC suffers heavy memory pressure produced by Sort in Spark, MURS suspends tasks with linear models in Sort and ensures enough memory space for WC. Almost all reduction time comes from the mitigated memory pressure because there are no suspended tasks in WC. Sort gets more memory space after WC completes, and has less memory pressure than Spark. 

%As the Sort application is split into three stages, only tasks in the second stage (this stage just prepares records for the next stage and writes records to disk) and the third stage (this is the actual stage that sorts the key) is the linear model. The function API in the first stage is \textit{flatMap}, which flats the result and provides an iterator, thus it belongs to the constant model. We find that these tasks with sub-linear and linear models suffer greater memory pressure, and all benefit from MURS. However, we can see that the constant model is also a little better, although the garbage collection time has a lower percentage in the original execution time—such as tasks in Grep and the first stage of Sort. This is due to the parallelism in writing to disk. As other tasks suffer heavy memory pressure, MURS stops some tasks and prevents them from writing shuffle data to disk. The accessibility of the disk can be faster, and those running tasks with the constant model can quickly complete their operation on disk. 

\subsection{Memory pressure with caching}

Some frameworks provide caching mechanism for in-memory computation, such as Spark and Flink. Although the in-memory caching data speed up the execution of a job, some works~\cite{bu:bloat, nguyen2015facade} show that it results in greater memory pressure because the caching data live along with the job. Tracing these data is expensive and there is less accessible memory space for execution.

PR is an example of iterative computations that caches important intermediate data in memory. The function API in the first stage of PR is \textit{groupByKey}, which groups all values of the particular key without aggregation. The result of \textit{groupByKey} will be cached in memory and used in subsequent iterations. Thus, the memory usage models of these tasks are linear. Thus, the memory usage models of tasks in Stage 2-3 are linear.  The caching data is alive in memory until the job is completed. Each iteration is implemented in a stage that contains several function APIs: \textit{join}, \textit{map}, and \textit{reduceByKey}. The function APIs \textit{map} and \textit{reduceByKey} are the same as that in Grep and WC. Although \textit{join} distinguishes the key, the function API here just processes the keys in one partition, which means it does not shuffle all keys. Thus, all data objects produced by \textit{join} are temporary data objects, and the sampler will classify it as the constant model. We submit PR along with WC, similarly to Section~\ref{sec:motivation}. The result is shown in Figure~\ref{fig:pr-wc-exec}. We should notice that, when WC job completes, PR is running in Stage 3. During the process of WC, the execution time of WC job is decreased by 28\%, but at the same time the execution time of PR job increased by 26\%. After WC is complete, the execution time of PR job can be decreased by 58\%.

\begin{figure}[!t]
\centering
\includegraphics[width=0.4\textwidth]{multitalent-exec.pdf}
%\vspace{-2mm}
\caption{The Execution Time of PR Job and WC Job}
%\vspace{-4mm}
\label{fig:pr-wc-exec}
\end{figure}

\begin{figure}[!t]
\centering
\includegraphics[width=0.4\textwidth]{pr-wc-gc.pdf}
%\vspace{-2mm}
\caption{The Execution and GC Time of Tasks in PR and WC}
%\vspace{-4mm}
\label{fig:pr-wc-gc}
\end{figure}

Before WC completes, tasks in PR and WC both result in memory pressure. However, tasks in PR belong to linear model, while tasks in WC belong to sub-linear model. MURS will suspend these tasks in PR to prevent heavy memory pressure, thus tasks in WC can have a lower execution time.

The execution time of the second stage of PR increases because tasks in PR are always classified to heavy tasks in this scene. In the second stage, PR caches all intermediate data in memory until the job completes. Then the memory manager requires many CPU cycles to trace the caching data objects; and less memory space is accessible for execution, which results in heavy garbage collection. While MURS intends to mitigate the heavy memory pressure, tasks in PR are always classified to heavy tasks because there are no other tasks belong to linear models. Thus, the waiting time of suspended tasks increases greatly although the garbage collection time decreases in Figure~\ref{fig:pr-wc-gc}, and the second stage in PR is even worse in MURS than in Spark for service. Fortunately, as WC completes early, subsequent stages  in MURS suffer much less memory pressure (the garbage collection can be decreased as much as 94\% in some tasks in Figure~\ref{fig:pr-wc-gc}), and performance improves more. 

\subsection{Avoidance of spill}

MURS sets the red value to avoid spill when memory pressure is heavy in these systems who provide spill. We take the evaluation in the last section, which runs PR and WC in the Spark for service, to measure the spill tasks. MURS estimates the size of required memory space for each task, and ensures that running tasks can complete with the remaining memory space. As the estimation is inaccurate to some extent, we find that fewer tasks will still spill in MURS, as shown in Table~\ref{table:spill}. There are no spill tasks in WC in MURS and the spill tasks in PR decreased from 32\% to 2.5\% in comparing to Spark.

\begin{table}[!t]
\small
\centering
\caption{Spill Tasks in MURS and Spark}
\begin{tabular}{| c | c | c | c | c | c | c | c |}

\hline
\multirow{2}{*}{} & \multirow{2}{*}{\textbf{App}} & \multicolumn{3}{| c |}{\textbf{Spill Percentage}} & \multicolumn{3}{| c |}{\textbf{Spill Size (MB)}} \\
\cline{3-8}
 & & total & spill & percent & min & mid & max \\
\hline
\multirow{2}{*}{Spark} & WC & 1000 & 91 & 9\% & 0 & 0 & 710 \\
\cline{2-8}
 & PR & 1500 & 480 & 32\% & 310 & 367 & 439 \\
\hline
\multirow{2}{*}{MURS} & WC & 1000 & 0 & 0\% & - & - & -  \\
\cline{2-8}
 & PR & 1500 & 37 & 2.5\% & 0 & 0 & 458 \\
\hline

\hline
\end{tabular}
%\vspace{-2mm}
 
%\vspace{-6mm}
\label{table:spill}
\end{table}

We should acknowledge that error exists in the avoidance of spill. The sampler in MURS counts two important metrics of one task: the percentage of processed records in total records, and the current allocated memory space for this task. We can quickly get the required memory space based on these two metrics. When memory pressure reaches the red value, we suspend parts of tasks and leave enough remaining memory space for running tasks. As the estimate is based on sampling, error exists in the estimated size of allocated memory space, especially when the value in some key-value pairs is a collection. Different records have different sizes because the number of values inside a collection is different, such as \textit{groupByKey} in PR. Some hot keys may result in substantial error. Thus, it can be accepted that there are even fewer spill tasks in MURS.

\subsection{Memory pressure in multi-launch}

Some systems for service are oversold and launches multiple tasks than the original configuration. When the service is idle, it has efficient memory usage. However, if the service comes to busy, the memory pressure is uncontrollable and the performance will degrade quickly when memory pressure is heavy. MURS is appropriate for this problem as it keeps the advantage in light memory pressure but prevents memory pressure increasing fast in heavy memory pressure. We choose WC here and two datasets to compare the impact of multi-launch MURS with Spark. We submit WC independently here to clearly distinguish the light and heavy memory pressure, and MURS can also work as mentioned in Section~\ref{sec:desgin}. WC processes data as (\textit{K}, \textit{V}), while \textit{K} means the words in the dataset. Thus, the number of words will decide the size of shuffle buffer, and more words will result in more memory pressure. As shown in Figure~\ref{fig:subfig:wc-million}, the GC time can be increased to 31.8\%, but the reduction of execution time is 11.6\% when the number of words is 1 million; the reduction of execution time is 22.9\% and the GC time can be 45.2\% when the number of words increases to 100 million in Figure~\ref{fig:subfig:wc-billion}.

%Although MURS is designed for data processing systems for service, we find it can also work well in batch processing when all running tasks have the same model. We choose WC here and two datasets to compare the impact of multi-launch in different memory pressures. WC process data as (\textit{K}, \textit{V}), while \textit{K} means the words in the dataset. Thus, the number of words will decide the size of shuffle buffer, and more words will result in more memory pressure. As shown in Figure~\ref{fig:subfig:wc-million}, the GC time can be increased to 31.8\%, but the reduction of execution time is 11.6\% when the number of words is 1 million; the reduction of execution time is 22.9\% and the GC time can be 45.2\% when the number of words increases to 1 billion in Figure~\ref{fig:subfig:wc-billion}.

\begin{figure}[!t]
\centering
\subfigure[Light memmory pressure]{
\label{fig:subfig:wc-million}
\includegraphics[width=0.231\textwidth]{wc-million.pdf}}
\hspace{-1.3ex}
\subfigure[Heavy memory pressure]{
\label{fig:subfig:wc-billion}
\includegraphics[width=0.231\textwidth]{wc-billion.pdf}}
\label{fig:wc-result}
%\vspace{-2mm}
\caption{Multi-launch with MURS when service is busy}
%\vspace{-4mm}
\end{figure}

Light memory pressure usually means that memory space will suffer less allocation and reclamation. Multi-launch increases the parallelism as well as the memory pressure, which properly increase the frequency of memory allocation and reclamation. This improves the efficiency of memory usage. Thus the services work well with multi-launch in Figure~\ref{fig:subfig:wc-million}. MURS remains the advantages in this scene. However, when the words increases to 100 million, memory pressure is high and the throughout is only 25.6\%. Frequent garbage collection degrades the performance quickly. MURS will suspend numbers of tasks to prevent the increase of memory pressure, thus both execution time and garbage collection time decrease in Figure~\ref{fig:subfig:wc-billion}.



















%%%%\begin{comment}

\subsection{Batch Processing}

\subsubsection{Impcat of Shuffle Function}
WordCount in Spark benchmark is an original MapReduce application. WC has two stages: the first stage reads data from HDFS and writes shuffle buffers; the second stage reads data from shuffle buffers and then function \textit{reduceByKey} is used to get the result. Shuffle buffers process data as (\textit{Key},\textit{Value}) while \textit{Key} means the words in dataset. Thus the number of words will decide the size of shuffle buffers, and more words will result in more memory pressure. As shown in Figure~\ref{fig:subfig:wc-million}, the GC time can be increased to 31.8\% but the reduction of execution time is 11.6\% when the number of words is 1 million; the reduction of execution time is 22.9\% and the GC time can be 45.2\% when the number of words increases to 1 billion in Figure~\ref{fig:subfig:wc-billion}.

When the size of words is small, the shuffle buffers will have less keys. The size of shuffle buffers will not result in heavy memory pressure which can also be proved by the light garbage collection. MURS provides multi launch to increase the memory pressure and parallelism of job, the running tasks is 1.5x than Spark. Thus the execution time decreases but the garbage collection time increases in Figure~\ref{fig:subfig:wc-million}. However, when the words increases to 1 billion, memory pressure can be high and the throughout is only 25.6\%. MURS will stop numbers of tasks to prevent the increase of memory pressure and both execution time and garbage collection time will decrease in Figure~\ref{fig:subfig:wc-billion}. 

\subsubsection{Impcat of Caching and Shuffle Function}
PageRank(PR) will firstly cache important intermediate data of function \textit{groupByKey} in memory. Each of the appending several iterations is a stage in the job. The main functions in each stage are \textit{join} and \textit{reduceByKey}. The function \textit{join} not belongs to shuffle function in PR. We test 10 iterations here and the result is shown in Figure~\ref{fig:pr-stagetime}. While the first stage just read data from HDFS, the memory pressure is light. After stages will suffer from caching data in memory. The size of caching data in each node is 4.1GB-5.3GB. Different stage has different memory pressure, the max execution time has the speed up of 2.1x to the min in Spark. Different memory pressure and tasks result in different performance of MURS. The best reduction of execution time can be 56\% and the average reduction is 44\% for computing stages.

\begin{figure}[!t]
\centering
\includegraphics[width=0.45\textwidth]{pr-stagetime.pdf}
\vspace{-2mm}
\caption{The Result of Each Stages in PageRank}
\vspace{-2mm}
\label{fig:pr-stagetime}
\end{figure}

The caching data cost constant memory space because the caching data has same lifetime in the job. While less execution memory space is accessible for shuffle operation, the same data objects will result in worse memory pressure and more frequently garbage collection. The improvement of PageRank in MURS is benefit from three points: GC reduction, spill avoidance  and stopping tasks.

\begin{figure}[!t]
\centering
\includegraphics[width=0.35\textwidth]{pr-runtasks.pdf}
\vspace{-2mm}
\caption{The Number of Running Tasks in MURS}
\vspace{-2mm}
\label{fig:pr-runtasks}
\end{figure}

\textbf{Stopping tasks} The core scheduling in MURS is stopping parts of tasks to prevent heavy memory pressure, thus we firstly analysis the count of running tasks as shown in Figure~\ref{fig:pr-runtasks}. The first stage has no caching data but some shuffle buffers in memory. The shuffle buffer will be live in JVM heap until write to disk, thus they result in some memory pressure and our scheduler stop two tasks accordingly. When the job runs in after stages, caching data will stay in memory until the job complete. The heavy memory pressure result in more stopped tasks.

\begin{figure}[!t]
\centering
\includegraphics[width=0.45\textwidth]{pr-gc.pdf}
\vspace{-2mm}
\caption{The GC of Total Tasks in PageRank}
\vspace{-4mm}
\label{fig:pr-gc}
\end{figure}

\textbf{GC reduction} Each task suffers from different garbage collection, as shown in Figure~\ref{fig:pr-gc}. We compare the minimum, median and maximum of total tasks. Each guideline in MURS is better than Spark. The median of garbage collection time in MURS has the speed up of 5.5x. The reduction of garbage collection can be more cleared with the peak execution memory of each task. Peak execution memory of each task is 457MB/533MB/868MB(min/mid/max) in MURS, and 369MB/493MB/597MB(min/mid/max) in Spark. As MURS stop some tasks to avoid the cost of memory space which will be occupied for a long time, other running tasks has more execution memory. While the same memory space is provided for less running tasks, the peak execution memory of each tasks will be high. If the peak execution memory is high, tasks can run without heavy memory pressure because less garbage collection will occur. After the running tasks release their occupied space, stopped tasks can also improve their peak execution memory. Thus the memory pressure in MURS is lighter than Spark, and more execution memory slow down the heavy garbage collection essentially.

We can also get the conclusion that the straggler can be avoided in some way. The max garbage collection time and the average time is much near in MURS but fluctuating serious in Spark. As the time of garbage collection makes up the important part of execution time, tasks with long period of garbage collection will have long execution time. If the execution time of one task exceeds the average time of completed tasks, we regard it as an straggler. Thus MURS will have low probability to produce the straggler.

\begin{table}[!t]
\small
\centering
\begin{tabular}{| c | c | c | c | c | c | c | c |}

\hline
\multirow{2}{*}{} & \multirow{2}{*}{Stage} & \multicolumn{3}{| c |}{Spill Percent} & \multicolumn{3}{| c |}{Spill Size (MB)} \\
\cline{3-8}
 & & total & spill & percent & min & mid & max \\
\hline
\multirow{2}{*}{Spark} & stage6 & 300 & 185 & 62\% & 0 & 370 & 480 \\
\cline{2-8}
 & stage1 & 300 & 52 & 17\% & 0 & 0 & 471 \\
\hline
\multirow{2}{*}{MURS} & stage6 & 300 & 0 & 0\% & - & - & -  \\
\cline{2-8}
 & stage1 & 300 & 1 & 1\% & 0 & 0 & 399 \\
\hline

\hline
\end{tabular}
\vspace{-2mm}
\caption{Spill Tasks in MURS and Spark} 
\vspace{-4mm}
\label{table:pr-spill}
\end{table}

\textbf{Spill avoidance} Spark has several spill tasks in each stage, and only one stage of MURS has spill tasks, as shown in Table~\ref{table:pr-spill}. No matter in the first stage or other stages, most tasks can avoid spilling in MURS although they would spill in Spark. The fundamental reason is also more execution memory. MURS has the spill computing algorithm to avoid spilling. After stopping tasks, remain space are enough to running tasks is just the goal of MURS. While multi launch is available, the spill computing algorithm is more functional to control the running tasks. We notice that spill also appear in MURS, there are two points here: 1) estimating the remaining using space of one running task is not usually accurate; 2) the available memory space of one thread in JVM is limited (1/2N at least and 1/N at most while N is the number of running threads in JVM). Spill can result in expensive disk IO, thus the performance can be better.% in MURS.

\subsection{Multi-tenant}

Spark JobServer can provide multi-tenant for Spark. We submit both WordCount and PageRank to the Spark JobServer to test the performance of MURS in multi-tenant. We set the number of iterations to be 3 in PageRank, when the WordCount completes, PageRank usually complete the second stage. The result is shown in Figure~\ref{fig:mul-exec}. The execution time of WordCount decreased 28\% but at the same time the execution time of PageRank increased 26\%. After WordCount is complete, the execution of PageRank can be decreased to 58\%. We should notice that, when the WordCount is complete the PageRank runs both in Stage 2.

In the first stage, tasks in PR and WC are both \textit{ShuffleMapTask} which result in memory pressure through the shuffle buffer in shuffle writer. However, the function APIs in PR is \textit{groupBykey} which is a non-aggregation, it increases the size of shuffle buffer for each processed record. The function APIs in WC is \textit{reduceByKey} which is an aggregation, it increases the size of shuffle buffer only when the \textit{K} of processed record has never appeared. Obviously, the tasks in PR belong to linear while the tasks in WC belong to sub-linear. Our scheduler will stop the tasks of PR to prevent the heavy memory pressure, thus the tasks of WC can have less execution time. The pity is that the waiting time of stopped tasks increased much. Fortunately, as WC completes early, the third stage (Stage 2) in MURS suffer much less memory pressure, and the performance improves more. This is much important in multi-tenant, not only the tasks with heavy influence on memory pressure can execute more quickly, but also these tasks with light influence can avoid the memory pressure. The service of all tenant, which means PR and WC, can be better.

\end{comment}


